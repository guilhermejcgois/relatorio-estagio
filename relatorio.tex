\documentclass{ufscar}

\begin{document}

\section{Descrição da Organização}
A Fundação de Estudos e Pesquisas Agrícolas e Florestais (FEPAF) foi fundada em Botucatu, no ano de 1980, com personalidade jurídica de direito privado e sem fins lucrativos. Sediada na Faculdade de Ciências Agronômicas (FCA) da UNESP (Univerisdade Estadual Paulista "Júlio de Mesquita Filho"), mais precisamente na Fazenda Experimental Lageado tem por finalidade apoiar programas de desenvolvimento sustentável (nas 3 esferas, econmômica, social e ambiental) da própria UNESP e também de outras instituições, através da articulação com estas.

Sua principal atividade é a administração de projetos de pesquisa realizados por equipes de docentes da FCA e de outras instituições relacionadas à agricultura, agroindústria, indústria de insumos e equipamentos agropecuários, preservação e recuperação ambiental. E, coerente com sua missão social, estes projetos contemplam bolsas de auxílio aos alunos de graduação e pós-graduação envolvidos nas pesquisas.

Algumas das atividades realizadas pela instituição estão ligadas a difusão e divulgação de conhecimento (ambientais, agronômicos, florestais e correlatos) através cursos universitários de especialização e extensão, simpósios, seminários e publicações técnicas, periódicas, monografias etc. Também com um viés dentro da academia, a FEPAF busca também a colaboração no aperfeiçoamento  dos cursos de pós-graduação da FCA e de outros unidades da UNESP, ou de qualquer outra instituição universitária que solicite seus serviços, além do desenvolvimento de pesquisas que atendam às necessidades tanto do setor público como no privado.

\section{Descrição do ambiente tecnológico}
Como trabalho remotamente, para desenvolver o projeto utilizo meu próprio ultrabook com as seguintes configurações:
\begin{itemize}
  \item \textbf{Modelo:} Lenovo Ideapad 14
  \item \textbf{Sistema Operacional:} Fedora 24 64-bit
  \item \textbf{Memória RAM:} 4GB
  \item \textbf{Processador:} Intel\textsuperscript{\textregistered} Core\texttrademark  i3-4010U CPU @ 1.70GHz x 4
  \item \textbf{HD:} 500GB
  \item \textbf{Bateria:} Duração média de 4h com brilho máximo
\end{itemize}
Falando em software, estão sendo utilizados os seguintes com respectivas versões:
\begin{itemize}
  \item Apache httpd v2.4.18
  \item PostgreSQL v9.5.4
  \item PostGIS v2.1.1
  \item PHP v5.6.25
  \item PhalconPHP v3.0.1
  \item LeafletJS v1.0.0-rc.3
  \item Mozilla Firefox 47.0
  \item Atom Editor v1.9.9
  \item Git v2.7.4
  \item NPM v2.15.9
  \item Bower v1.7.9
  \item React v15.3.2
\end{itemize}

\section{Relatório das atividades desenvolvidas pelo aluno}
\subsection{Estágio 1}
Nos primeiros dois meses o trabalho com o Sistema Informatizado de Gestão Ambiental do município de São Roque envolveu basicametne além de seu desenvolvimento também a elaboração conceitual de acordo com suas peculiariedades e necessidades. Tal elaboração inicial compreendeu todo um estudo a cerca de sistemas semelhantes e como operam, identificando pontos fortes e fracos para serem aproveitados ou evitados no projeto, visando sempre a necessidade do cliente como também os dados provenientes da equipe responsável por estes. Tais dados necessitam ser tratados por uma equipe separada pelo fato de serem dados específicos, geoespaciais de nível técnico da área ambiental, logo esses dados chegam prontos e apenas são inseridos no banco de dados para posterior exibição. Esses dados nós chamamos de camadas, as camadas são compostas por feições que representam características desta, sendo que uma camada representa um aspecto geográfico de um dado espaço, podendo representar por exemplo relevo, hidrografia, bairros, etc.

Na etapa de desenvolvimento trabalhamos em um produto que oferecesse funcionalidades básicas que possibilitassem o uso do sistema, que está sendo desenvolvido de maneira incremental em questão de funcionalidades. Começamos apenas exibindo o mapa do espaço geográfico
que queremos, depois juntamente com o início da programaação do lado servidor começamos a inserir e remover apenas uma camada para em seguida trabalhar com mais de uma. Em seguida foi desenvolvido um controle de interface que nos permite alterar o mapa do espaço geográfico (trocar entre um mapa colorido e um mapa em preto e branco por exemplo) e trabalhar na inserção e remoção de camadas no mapa, além de trabalhar a opacidade de uma camada em específico, visto que comos concluímos durante nossa pesquisa durante a fase da elaboração conceitual que podemos ter um cenário onde mais de uma camada se sobrepõem de tal maneira que uma oculta totalmente uma informação de outra.

A partir dessa primeira versão partimos para algo mais ousado, desenvolver um controle de interface que nos permitisse trabalhar com o empilhamento das camadas, então o problema de uma sobrepor a outra podemos resolver agora simplesmente pegando a camada sendo sobreposta e colocando-a no topo da pilha das camadas. Mas não foi apenas isso, começamos a adotar a React, uma biblioteca para construir e gerenciar componentes de interfaces desenvolvidas e mantidas pelo Facebook. No lado do servidor, começamos a trabalhar com metadados nas tabelas das camadas no banco de dados, uma vez que o nome da tabela e das colunas não podiam ser utilizados para exibição com o usuário, consequentemente nos levou a ter a necessidade de programar mais um serviço para expor esse novo recurso ao usuário através de nossa API.

O cronograma seguido pode ser conferido na tabela abaixo:

\begin{table}[]
\centering
\caption{Cronograma}
\label{my-label}
\begin{tabular}{|l|l|l|}
\hline
\textbf{Início} & \textbf{Fim} & \textbf{Atividades} \\ \hline
01/07 & 01/08 & \begin{tabular}[c]{@{}l@{}}Pesquisa de produtos relacionados; \\ Familiarização com tecnologias; \\ Definição de Requisitos e Product Backlog\end{tabular} \\ \hline
02/08 & 16/08 & \begin{tabular}[c]{@{}l@{}}Implementação inicial do back end; \\ Implementação da interface inicial com controles de zoom e camadas\end{tabular} \\ \hline
17/08 & 05/09 & \begin{tabular}[c]{@{}l@{}}Adoção da React;\\ Adoção do uso de metadados no banco de dados;\\ Implementação do controle de camadas ativas\end{tabular} \\ \hline
\end{tabular}
\end{table}

\section{Reflexão por parte do aluno sobre as principais contribuições ao projeto}
Como o projeto foi iniciado do zero comigo, minha contribuição está em todo o projeto. Toda a fundação do projeto está sendo feita por mim, desde definição da arquitetura até a interface, sendo elas construídas pensando na manutenibilidade do sistema como um todo, além ainda de procurar manter o código de fácil extensão e modificação (o \textit{framework} utilizado no \textit{backend} possibilita isso), entregar um código padronizado e de qualidade é uma preocupação. Também há documentação em código deixando mais claro seu funcionamento em alguns pontos para facilitar o entendimento do que aquele trecho de código está fazendo, pois há uma preocupação de minha parte para com quem venha a mexer em meu código depois de mim, ou mesmo que entre outro membro no projeto e até mesmo caso eu volto a ver o código depois de um tempo e lembrar o que o mesmo faz. Outra preocupação também está sendo quanto ao \textit{pipeline} do desenvolvimento da aplicação, buscando deixá-lo o simples de se entender e usar, melhorando a produtividade durante o processo de desenvolvimento.

\section{Relação dos principais conhecimentos obtidos nas disciplinas do curso e que foram de importância para o estágio}
Algumas disciplinas da graduação expõem conteúdos que foram e/ou estão sendo importantes no desenvolvimento do projeto no estágio:
\begin{enumerate}

  \item Algoritmos e Programação 1
  \begin{itemize}
    \item \textbf{Estruturas de controle:} por mais que um conceito básico, não deixou de ser importante no desenvolvimento do projeto, visto que estruturas de controle são usadas exaustivamente por todo o código do projeto.
  \end{itemize}

  \item Pesquisa Acadêmica em Computação
  \begin{itemize}
    \item Quando realizei a disciplina, meu trabalho semestral teve como tema a aplicação de banco de dados geoespaciais na agricultura, onde ao final da disciplina foi escrito um estudo bibliográfico sobre o tema, curiosamente é quase com o que estou trabalhando, então algumas coisas que pesquisei sobre banco de dados geoespaciais, como funcionam e como os dados são e de que tipo são guardados isso me poupou tempo para trabalhar com esses dados no projeto.
  \end{itemize}

  \item Programação Orientada a Objetos
  \begin{itemize}
    \item \textbf{Sintaxe:} como C++ foi a primeira linguagem de programação orientada a objetos trabalhada no curso, e além disso é uma das linguagens nos quais PHP é baseada, a sintaxe entre ambas é muito parecido, o que diminui consideravelmente a minha curva de aprendizagem com essa nova linguaguem, já que PHP era algo totalmente novo para mim.
    \item \textbf{Conceitos de orientação a objetos (OO):} como PHP também é uma linguagem OO, assim como C++ e Java, alguns conceitos de OO se fazem presentes pelo código do lado servidor da aplicação, como classes e encapsulamento, para citar apenas dois.
  \end{itemize}

  \item Engenharia de Software 1
  \begin{itemize}
    \item \textbf{Requisitos:} na fase inicial do projeto, foram levantados alguns requisitos (funcionais e não funcionais), praticas na disciplina facilitaram na identificação destes para o projeto, mesmo o escopo do projeto sendo bastante diferente do trabalho na disciplina.
  \end{itemize}

  \item Banco de Dados
  \begin{itemize}
    \item \textbf{Structured Query Language (SQL):} por mais que boa parte dos dados que trabalho são diferentes, o básico da sintaxe e funcionamento é o mesmo, tando instruções de definição como manipulação de dados estão sendo muito importantes, até mesmo consultas aninhadas estão sendo usadas.
  \end{itemize}

  \item Laboratório de Banco de Dados
  \begin{itemize}
    \item \textbf{Metadados:} no projeto estamos trabalhando com metadados de tabelas, além de uso de \textit{triggers} para validar inserção de dados em tabelas, futuramente podemos ainda trabalhar com indexação e otimização de busca também.
  \end{itemize}

  \item Desenvolvimento para Web
  \begin{itemize}
    \item \textbf{Programação do lado cliente:} a maior parte do conteúdo da disciplina sobre programação no lado cliente, com HTML, CSS e JavaScript é fortemente utilizada, até mesmo jQuery e requisições via AJAX.
    \item \textbf{Arquitetura MVC:} a arquitetura MVC (Model View Controller) apresentada na disciplina também pode ser aproveitada no projeto, pois ajudou na separação e organização do código, mesmo que na aula a arquitetura tenha sido vista com tecnologia Java.
  \end{itemize}

  \item Sistemas Distribuídos
  \begin{itemize}
    \item \textbf{Web Services:} web services (WS) é outra tecnologia que também utilizamos bastante, tanto que os recursos oferecidos pelos WS são consumidos por requisições AJAX.
  \end{itemize}

  \item Interface Humano-Computador
  \begin{itemize}
    \item Alguns dos fundamentos discutidos em sala pela professora foram levados em conta na hora de construir a interface, como affordance, comunicabilidade e usabilidade.
  \end{itemize}

\end{enumerate}

\section{Reflexão sobre as dificuldades enfrentadas pelo estagiário na organização}
No começo do projeto tive um pouco de dificuldades em configurar o projeto no apache, na verdade o problema foi na hora de mudar a estrutura do diretório do projeto localmente devido a um arquivo de configuração que basicamente faz a reescrita das URLs que chegam para a aplicação através do Apache. Também durante o primeiro mês, como o estágio é remoto, mesmo separando e distribuindo as 30 horas semanais ao longo da semana tive um pouco de receio quanto a minha produtividade, as vezes me perguntava se eu realmente estava sendo produtivo o suficiente para com o projeto, mas ao longo dos encontros quinzenais, fui me aquietando devido ao feedback positivo do meu supervisor na organização.
Achei que pudesse ter dificuldades de comunicação com meu supervisor e eventualmente com a equipe responsável por me passar os dados que trabalharei na exibição do projeto, já que eles não são pessoas técnicas da minha área mas não tive dificuldades, inclusive como recomendação do meu supervisor realizei 2 \textit{workshop} sobre o software QGIS que é onde são gerados os dados que me enviam, sendo um no começo mês de agosto e outro nesse mês passado, que ao meu ver contribuiram não só em conhecimento como também na parte de comunicação devido a termos técnicos e também na hora de trabalhar na arquitetura de informação do sistema, usando de modo apropriado palavras técnicas da área deles.

\section{Relação de tópicos que poderiam ser estudados no curso de Computação e que foram necessários no estágio.}
Os tópicos a seguir são alguns dos tópicos necessário no estágio que acredito que poderiam ter sido abordados no curso de Computação:
\begin{itemize}
  \item \textbf{Git:} tecnologia para versionamento popular que merece atenção, pois permite além de fazer o controle das versões do código torna fácil realizar \textit{rollback} na base de código em algum ponto especifíco do tempo.
\end{itemize}

\end{document}
