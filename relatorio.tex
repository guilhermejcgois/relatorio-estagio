\documentclass{ufscar}

\begin{document}

\section{Descrição da Organização}
(Máximo: 2 páginas)

Conteúdo: histórico e ambiente de negócios em  que a empresa opera (contexto, o setor econômico, os principais produtos, concorrentes, fornecedores, consumidores, etc)

\section{Descrição do ambiente tecnológico}
Como trabalho remotamente, para desenvolver o projeto utilizo meu próprio ultrabook com as seguintes configurações:
\begin{itemize}
  \item \textbf{Modelo:} Lenovo Ideapad 14
  \item \textbf{Sistema Operacional:} Fedora 24 64-bit
  \item \textbf{Memória RAM:} 4GB
  \item \textbf{Processador:} Intel\textsuperscript{\textregistered} Core\texttrademark  i3-4010U CPU @ 1.70GHz x 4
  \item \textbf{HD:} 500GB
  \item \textbf{Bateria:} Duração média de 4h com brilho máximo
\end{itemize}
Falando em software, estão sendo utilizados os seguintes com respectivas versões:
\begin{itemize}
  \item Apache httpd v2.4.18
  \item PostgreSQL v9.5.4
  \item PostGIS v2.1.1
  \item PHP v5.6.25
  \item PhalconPHP v3.0.1
  \item LeafletJS v1.0.0-rc.3
  \item Mozilla Firefox 47.0
  \item Atom v1.9.9
  \item Git v2.7.4
  \item NPM v2.15.9
  \item Bower v1.7.9
  \item Gulp v1.2.2
\end{itemize}

\section{Relatório das atividades desenvolvidas pelo aluno}
\subsection{Estágio 1}
- Descrição do conteúdo (detalhamento)
- Cronograma das atividades
- Resultados atingidos


\section{Reflexão por parte do aluno sobre as principais contribuições ao projeto}
Como o projeto foi iniciado do zero comigo, minha contribuição está em todo o projeto. Toda a fundação do projeto está sendo feita por mim, desde definição da arquitetura até a interface, sendo elas construídas pensando na manutenibilidade do sistema como um todo, além ainda de procurar manter o código de fácil extensão e modificação (o \textit{framework} utilizado no \textit{backend} possibilita isso), entregar um código padronizado e de qualidade é uma preocupação. Também há documentação em código deixando mais claro seu funcionamento em alguns pontos para facilitar o entendimento do que aquele trecho de código está fazendo, pois há uma preocupação de minha parte para com quem venha a mexer em meu código depois de mim, ou mesmo que entre outro membro no projeto e até mesmo caso eu volto a ver o código depois de um tempo e lembrar o que o mesmo faz.

\section{Relação dos principais conhecimentos obtidos nas disciplinas do curso e que foram de importância para o estágio}
Algumas disciplinas da graduação expõem conteúdos que foram e/ou estão sendo importantes no desenvolvimento do projeto no estágio:
\begin{enumerate}

  \item Algoritmos e Programação 1
  \begin{itemize}
    \item \textbf{Estruturas de controle:} por mais que um conceito básico, não deixou de ser importante no desenvolvimento do projeto, visto que estruturas de controle são usadas exaustivamente por todo o código do projeto.
  \end{itemize}

  \item Pesquisa Acadêmica em Computação
  \begin{itemize}
    \item Quando realizei a disciplina, meu trabalho semestral teve como tema a aplicação de banco de dados geoespaciais na agricultura, onde ao final da disciplina foi escrito um estudo bibliográfico sobre o tema, curiosamente é quase com o que estou trabalhando, então algumas coisas que pesquisei sobre banco de dados geoespaciais, como funcionam e como os dados são e de que tipo são guardados isso me poupou tempo para trabalhar com esses dados no projeto.
  \end{itemize}

  \item Programação Orientada a Objetos
  \begin{itemize}
    \item \textbf{Sintaxe:} como C++ foi a primeira linguagem de programação orientada a objetos trabalhada no curso, e além disso é uma das linguagens nos quais PHP é baseada, a sintaxe entre ambas é muito parecido, o que diminui consideravelmente a minha curva de aprendizagem com essa nova linguaguem, já que PHP era algo totalmente novo para mim.
    \item \textbf{Conceitos de orientação a objetos (OO):} como PHP também é uma linguagem OO, assim como C++ e Java, alguns conceitos de OO se fazem presentes pelo código do lado servidor da aplicação, como classes e encapsulamento, para citar apenas dois.
  \end{itemize}

  \item Engenharia de Software 1
  \begin{itemize}
    \item \textbf{Requisitos:} na fase inicial do projeto, foram levantados alguns requisitos (funcionais e não funcionais), praticas na disciplina facilitaram na identificação destes para o projeto, mesmo o escopo do projeto sendo bastante diferente do trabalho na disciplina.
  \end{itemize}

  \item Banco de Dados
  \begin{itemize}
    \item \textbf{Structured Query Language (SQL):} por mais que boa parte dos dados que trabalho são diferentes, o básico da sintaxe e funcionamento é o mesmo, tando instruções de definição como manipulação de dados estão sendo muito importantes, até mesmo consultas aninhadas estão sendo usadas.
  \end{itemize}

  \item Laboratório de Banco de Dados
  \begin{itemize}
    \item \textbf{Metadados:} no projeto estamos trabalhando com metadados de tabelas, além de uso de \textit{triggers} para validar inserção de dados em tabelas, futuramente podemos ainda trabalhar com indexação e otimização de busca também.
  \end{itemize}

  \item Desenvolvimento para Web
  \begin{itemize}
    \item \textbf{Programação do lado cliente:} a maior parte do conteúdo da disciplina sobre programação no lado cliente, com HTML, CSS e JavaScript é fortemente utilizada, até mesmo jQuery e requisições via AJAX.
    \item \textbf{Arquitetura MVC:} a arquitetura MVC (Model View Controller) apresentada na disciplina também pode ser aproveitada no projeto, pois ajudou na separação e organização do código, mesmo que na aula a arquitetura tenha sido vista com tecnologia Java.
  \end{itemize}

  \item Sistemas Distribuídos
  \begin{itemize}
    \item \textbf{Web Services:} web services (WS) é outra tecnologia que também utilizamos bastante, tanto que os recursos oferecidos pelos WS são consumidos por requisições AJAX.
  \end{itemize}

  \item Interface Humano-Computador
  \begin{itemize}
    \item Alguns dos fundamentos discutidos em sala pela professora foram levados em conta na hora de construir a interface, como affordance, comunicabilidade e usabilidade.
  \end{itemize}

\end{enumerate}

\section{Reflexão sobre as dificuldades enfrentadas pelo estagiário na organização}
No começo do projeto tive um pouco de dificuldades em configurar o projeto no apache, na verdade o problema foi na hora de mudar a estrutura do diretório do projeto localmente devido a um arquivo de configuração que basicamente faz a reescrita das URLs que chegam para a aplicação através do Apache. Também durante o primeiro mês, como o estágio é remoto, mesmo separando e distribuindo as 30 horas semanais ao longo da semana tive um pouco de receio quanto a minha produtividade, as vezes me perguntava se eu realmente estava sendo produtivo o suficiente para com o projeto, mas ao longo dos encontros quinzenais, fui me aquietando devido ao feedback positivo do meu supervisor na organização.
Achei que pudesse ter dificuldades de comunicação com meu supervisor e eventualmente com a equipe responsável por me passar os dados que trabalharei na exibição do projeto, já que eles não são pessoas técnicas da minha área mas não tive dificuldades, inclusive como recomendação do meu supervisor realizei 2 \textit{workshop} sobre o software QGIS que é onde são gerados os dados que me enviam, sendo um no começo mês de agosto e outro nesse mês passado, que ao meu ver contribuiram não só em conhecimento como também na parte de comunicação devido a termos técnicos.

\section{Relação de tópicos que poderiam ser estudados no curso de Computação e que foram necessários no estágio.}
Os tópicos a seguir são alguns dos tópicos necessário no estágio que acredito que poderiam ter sido abordados no curso de Computação:
\begin{itemize}
  \item \textbf{Git:} tecnologia para versionamento popular que merece atenção, pois permite além de fazer o controle das versões do código torna fácil realizar \textit{rollback} na base de código em algum ponto especifíco do tempo.
  \item \textbf{:}
\end{itemize}

\end{document}
