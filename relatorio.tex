\documentclass{ufscar}

\usepackage{float}
\restylefloat{table}

\begin{document}

\section{Descrição da Organização}
A Fundação de Estudos e Pesquisas Agrícolas e Florestais (FEPAF) foi fundada em Botucatu, no ano de 1980, com personalidade jurídica de direito privado e sem fins lucrativos. Sediada na Faculdade de Ciências Agronômicas (FCA) da UNESP (Univerisdade Estadual Paulista "Júlio de Mesquita Filho"), mais precisamente na Fazenda Experimental Lageado tem por finalidade apoiar programas de desenvolvimento sustentável (nas 3 esferas, econmômica, social e ambiental) da própria UNESP e também de outras instituições, através da articulação com estas.

Sua principal atividade é a administração de projetos de pesquisa realizados por equipes de docentes da FCA e de outras instituições relacionadas à agricultura, agroindústria, indústria de insumos e equipamentos agropecuários, preservação e recuperação ambiental. E, coerente com sua missão social, estes projetos contemplam bolsas de auxílio aos alunos de graduação e pós-graduação envolvidos nas pesquisas.

Algumas das atividades realizadas pela instituição estão ligadas a difusão e divulgação de conhecimento (ambientais, agronômicos, florestais e correlatos) através cursos universitários de especialização e extensão, simpósios, seminários e publicações técnicas, periódicas, monografias etc. Também com um viés dentro da academia, a FEPAF busca também a colaboração no aperfeiçoamento  dos cursos de pós-graduação da FCA e de outros unidades da UNESP, ou de qualquer outra instituição universitária que solicite seus serviços, além do desenvolvimento de pesquisas que atendam às necessidades tanto do setor público como no privado.

\section{Descrição do ambiente tecnológico}
Como trabalho remotamente, para desenvolver o projeto utilizo meu próprio ultrabook com as seguintes configurações:
\begin{itemize}
  \item \textbf{Modelo:} Lenovo Ideapad 14
  \item \textbf{Sistema Operacional:} Fedora 24 64-bit
  \item \textbf{Memória RAM:} 4GB
  \item \textbf{Processador:} Intel\textsuperscript{\textregistered} Core\texttrademark  i3-4010U CPU @ 1.70GHz x 4
  \item \textbf{HD:} 500GB
  \item \textbf{Bateria:} Duração média de 4h com brilho máximo
\end{itemize}
Falando em software, estão sendo utilizados os seguintes com respectivas versões:
\begin{itemize}
  \item Apache httpd v2.4.18
  \item PostgreSQL v9.5.4
  \item PostGIS v2.1.1
  \item PHP v5.6.25
  \item PhalconPHP v3.0.1
  \item LeafletJS v1.0
  \item Mozilla Firefox 47.0
  \item Atom Editor v1.9.9
  \item Git v2.7.4
  \item NPM v2.15.9
  \item Bower v1.7.9
  \item React v15.3.2
  \item GulpJS v3.9.1
  \item SASS 3.4.22
\end{itemize}

\section{Relatório das atividades desenvolvidas pelo aluno}
\subsection{Estágio 2}
Continuando o desenvolvimento do Sistema Informatizado de Gestão Ambiental do município de São Roque, foi adotado o uso de um automatizador de tarefas, GulpJS, para tornar a transpilação do código escrito com a React e também com a SASS (um pré-processador de CSS) automática, acontecento assim tão logo quanto as alterações nos arquivos sejam salvas. Também foi construído a ferramenta responsável pelo controle da opacidade de um camada, \it{feature} essa que tinha sido removida devido a adição do controle de empilhamento das camadas, onde ela atualmente se encontra disponível para uso.

Após finalizar a ferramenta de opacidade, foi iniciado o desenvolvimento da ferramenta de busca nas camadas, essa ferramenta de busca trabalha com dados comuns (inteiro, string, ponto flutuante, etc.) e não dados geográficos. Os dados inseridos pelo usuário são comparados com os dados das feições da camada em questão, destacando as camadas que conferem com os dados fornecidos pelo usuário dos que não conferem. Os dados disponíveis nas feições eram dados numéricos e strings e devido a natureza dos tipos de filtros que é possível fazer foi dividido em duas etapas. Outra \textit{feature} interessante é a recuperação dessas informações de uma feição, que ao clicar na feição o usuário pode conferir essas características através de um \textit{popup}, que se fecha ao usuário sair para fora dos limites dessa feição.

Foram feitas melhorias relacionadas a interface, tanto na interface apresentada ao usuário de fato como em sua composição estrutural, onde a componentização de seus componentes foi feita, basicamente deixando cada componente em seu arquivo próprio. Se tratando do que o usuário de fato vê e interaje, foram feitas algumas melhorias em questão de posicionamento e tamanho dos controles existentes, fechar a visualização do mapa apenas no município de São Roque já que não haverão dados a serem mostrados fora de seu limite, além de adicionarmos um controle que permite o rastreio das coordenadas geográficas a partir da posição do mouse sobre o mapa. Foi feita a adaptação de um plugin do Leaflet que realiza o cálculo de medidas (como distância entre dois pontos e área) para que este se adequasse a nossa interface, que subistitui a dele próprio e desacoplando os cálculos que ele pode fazer, no caso da distância e área, e sempre finalizava o desenho do que ia ser calculado com o duplo clique, então dois pontos levavam ao cálculo da distância entre eles, de 3 pontos em diante ele compreendia o desenho como um polígono e então calculava sua área e perímetro, e precisávamos que esses cálculos estivessem em controles separados.

As duas últimas \textit{features} da primeira versão do sistema foram implementadas, a impressão e o relatório do mapa. No caso da impressão, era necessário apenas imprimir o mapa tal como estava sendo visualizado, porém sem os controles de interação com o mapa. No caso do relatório, ele é feito apenas com a camada mais ao topo da pilha, listando todos os dados das feições da camada. Ambas as \textit{features} necessitam que o estado atual do mapa seja salvo no servidor, o que levou a necessidade da criação de serviços para salvar e recuperar esses estados, pois precisamos visualizarmos em aba separada o que será impresso para o usuário.

O cronograma pode ser conferido na tabela abaixo:

\begin{table}[H]
\centering
\caption{Cronograma}
\label{my-label}
\begin{tabular}{|l|l|l|}
\hline
\textbf{Início} & \textbf{Fim} & \textbf{Atividades} \\ \hline
06/09 & 20/09 & \begin{tabular}[c]{@{}l@{}}Adoção do GulpJS e do SASS; \\ Reimplementação da ferramenta de opacidade; \\ Implementação da busca por dados não numéricos \\ Exibir as características de uma feição para o usuário\end{tabular} \\ \hline
21/09 & 07/10 & \begin{tabular}[c]{@{}l@{}}Implementação da busca por dados numéricos; \\ Implementação do rastreio das coordenadas do mouse\end{tabular} \\ \hline
10/10 & 04/11 & \begin{tabular}[c]{@{}l@{}}Implementação das ferramentas de medidas;\\ Implementação da impressão;\\ Implementação do relatório\end{tabular} \\ \hline
\end{tabular}
\end{table}

\section{Reflexão por parte do aluno sobre as principais contribuições ao projeto}
Com o aumento considerado da base de código, foi adotado o uso de um automatizador de tarefas, o GulpJS. Conforme cresce o número de componentes da interface, usando GulpJS podemos definir uma tarefa para tratar os componentes já existentes e os novos que surgirem. Além de outros processos como minificação (técnica empregada para comprimir arquivos CSS e JavaScript) e \textit{lint} (análise estática de código).

Ao final do estágio, busca-se deixar um \textit{workflow} bem definido para manter a qualidade do código durante todo o processo de desenvolvimento e melhorar a produtividade deste, desde o preparo do ambiente de desenvolvimento até a preparação do fonte para ser utilizado em produção.

\section{Relação dos principais conhecimentos obtidos nas disciplinas do curso e que foram de importância para o estágio}
Algumas disciplinas da graduação expõem conteúdos que foram e/ou estão sendo importantes no desenvolvimento do projeto no estágio:
\begin{enumerate}

  \item Algoritmos e Programação 1
  \begin{itemize}
    \item \textbf{Estruturas de controle:} por mais que um conceito básico, não deixou de ser importante no desenvolvimento do projeto, visto que estruturas de controle são usadas exaustivamente por todo o código do projeto.
  \end{itemize}

  \item Pesquisa Acadêmica em Computação
  \begin{itemize}
    \item Quando realizei a disciplina, meu trabalho semestral teve como tema a aplicação de banco de dados geoespaciais na agricultura, onde ao final da disciplina foi escrito um estudo bibliográfico sobre o tema, curiosamente é quase com o que estou trabalhando, então algumas coisas que pesquisei sobre banco de dados geoespaciais, como funcionam e como os dados são e de que tipo são guardados isso me poupou tempo para trabalhar com esses dados no projeto.
  \end{itemize}

  \item Programação Orientada a Objetos
  \begin{itemize}
    \item \textbf{Sintaxe:} como C++ foi a primeira linguagem de programação orientada a objetos trabalhada no curso, e além disso é uma das linguagens nos quais PHP é baseada, a sintaxe entre ambas é muito parecido, o que diminui consideravelmente a minha curva de aprendizagem com essa nova linguaguem, já que PHP era algo totalmente novo para mim.
    \item \textbf{Conceitos de orientação a objetos (OO):} como PHP também é uma linguagem OO, assim como C++ e Java, alguns conceitos de OO se fazem presentes pelo código do lado servidor da aplicação, como classes e encapsulamento, para citar apenas dois.
  \end{itemize}

  \item Engenharia de Software 1
  \begin{itemize}
    \item \textbf{Requisitos:} na fase inicial do projeto, foram levantados alguns requisitos (funcionais e não funcionais), praticas na disciplina facilitaram na identificação destes para o projeto, mesmo o escopo do projeto sendo bastante diferente do trabalho na disciplina.
  \end{itemize}

  \item Banco de Dados
  \begin{itemize}
    \item \textbf{Structured Query Language (SQL):} por mais que boa parte dos dados que trabalho são diferentes, o básico da sintaxe e funcionamento é o mesmo, tando instruções de definição como manipulação de dados estão sendo muito importantes, até mesmo consultas aninhadas estão sendo usadas.
  \end{itemize}

  \item Laboratório de Banco de Dados
  \begin{itemize}
    \item \textbf{Metadados:} no projeto estamos trabalhando com metadados de tabelas, além de uso de \textit{triggers} para validar inserção de dados em tabelas, futuramente podemos ainda trabalhar com indexação e otimização de busca também.
  \end{itemize}

  \item Desenvolvimento para Web
  \begin{itemize}
    \item \textbf{Programação do lado cliente:} a maior parte do conteúdo da disciplina sobre programação no lado cliente, com HTML, CSS e JavaScript é fortemente utilizada, até mesmo jQuery e requisições via AJAX.
    \item \textbf{Arquitetura MVC:} a arquitetura MVC (Model View Controller) apresentada na disciplina também pode ser aproveitada no projeto, pois ajudou na separação e organização do código, mesmo que na aula a arquitetura tenha sido vista com tecnologia Java.
  \end{itemize}

  \item Sistemas Distribuídos
  \begin{itemize}
    \item \textbf{Web Services:} web services (WS) é outra tecnologia que também utilizamos bastante, tanto que os recursos oferecidos pelos WS são consumidos por requisições AJAX.
  \end{itemize}

  \item Interface Humano-Computador
  \begin{itemize}
    \item Alguns dos fundamentos discutidos em sala pela professora foram levados em conta na hora de construir a interface, como affordance, comunicabilidade e usabilidade.
  \end{itemize}

\end{enumerate}

\section{Reflexão sobre as dificuldades enfrentadas pelo estagiário na organização}
Entender como um \textit{plugin} funciona e adaptá-lo a sua própria necessidade se mostrou um grande desafio, cheguei até mesmo me perguntar se não seria mais fácil fazer do zero. Muitas vezes a documentação desses \textit{plugins} não é muito clara no que diz respeito a como as coisas funcionam, aí você precisar alterar algumas coisas mas não sabe por onde começa, levou tempo e paciência adaptar o controle de medidas. A impressão da página do sistema também não foi algo amigável de se fazer, pois devido a como o Leaflet trabalha a impressão não saia centralizada, nem a documentação me ajudou a resolver esse problema, mas boa parte do problema foi entender os eventos de impressão do navegador, e por final ficou sendo necessário abrir uma exibição da impressão sem mandar imprimir automaticamente.

\section{Relação de tópicos que poderiam ser estudados no curso de Computação e que foram necessários no estágio.}
Os tópicos a seguir são alguns dos tópicos necessário no estágio que acredito que poderiam ter sido abordados no curso de Computação:
\begin{itemize}
  \item \textbf{Git:} tecnologia para versionamento popular que merece atenção, pois permite além de fazer o controle das versões do código torna fácil realizar \textit{rollback} na base de código em algum ponto especifíco do tempo.
  \item \textbf{Impressão de páginas web:} o que se vê no navegador nem sempre fica do mesmo jeito no papel na hora da impressão, apanhei bastante para fazer essa \textit{feature}, isso porquê era para ser feita apenas em um tipo e modelo de folha (A4 em paisagem).
  \item \textbf{Automatizadores de tarefas:} independente da linguagem e ambiente, é visível a agilidade que essas ferramentas trazem, acredito que se nós alunos tivéssemos algum conhecimento sobre isso podíamos melhorar a  qualidade de nossos trabalhos e projetos práticos, que aliás é o que mais toma nosso tempo durante toda a graduação.
\end{itemize}

\end{document}
